\documentclass[a4paper,12pt]{article}

\usepackage[utf8]{inputenc}
\usepackage{polski}
\usepackage{fullpage}
\usepackage{hyperref}

\title{Wyznaczanie reprezentacji preferencji uniwersalych}
\author{Bartosz Górski \and Marcin Kaczyński \and Paweł Sokołowski}

\begin{document}

\maketitle

\section{Opis zadania} 

Szablon do wypełnienia. Jak ktoś chce napisać jakiś konkretny rozdział to niech się przy nim wpisze - może być w komentarzu. 
Jak nie to się później rozdzieli zadania.\\

Przykład wykorzystania bibliografii\cite{fst}. W bibliografii wpisy są z jakiegoś mojego wcześniejszego raportu, będą zmienione. Kompilacja:

\begin{itemize}
\item pdflatex raport (chyba musi być)
\item bibtex raport
\item pdflatex raport (x2)
\end{itemize}

Jak ktoś nie umie pisać w texie, to niech prześle plain text w notaniku.

\section{Założenia}
\section{Dane wejściowe i wyjściowe}
\section{Szczegóły implementacji}
\section{Wyniki}
\subsection{Wyniki podstawowe}

Tutaj będzie opis wyników dla danych z artykułu.

\subsection{Wyniki dla złożonych danych}

A tutaj wyniki dla normalnych danych. Na pewno musi być zbiór z samochodami, ten który był na którymś lab.

\section{Wnioski}

\appendix
\section{Podręcznik użytkownika}

:] bez komentarza

\bibliographystyle{plain}
\bibliography{bibliografia}

\end{document}
